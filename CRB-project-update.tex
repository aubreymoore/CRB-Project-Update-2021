\documentclass[12pt,letterpaper,english,bibliography=totocnumbered,abstract=on]{scrartcl}

\usepackage{indentfirst}
\usepackage{appendix}
\usepackage{fullpage}
%\usepackage{subfiles}
\usepackage[T1]{fontenc}
\usepackage[latin9]{inputenc}
\usepackage{color}
\usepackage{babel}
\usepackage{verbatim}
\usepackage[unicode=true,pdfusetitle,
bookmarks=true,bookmarksnumbered=false,bookmarksopen=false,
breaklinks=true,pdfborder={0 0 0},pdfborderstyle={},backref=false,colorlinks=true]
{hyperref}
\hypersetup{linkcolor=blue,citecolor=blue,urlcolor=blue}

\usepackage{booktabs}
\usepackage{multirow}
\usepackage{adjustbox}
\usepackage{threeparttable}
\usepackage[table]{xcolor}
\usepackage{csquotes}
\usepackage{soul} % for hiliting text: \hl
% old style is authoryear
\usepackage[backend=biber, style=numeric, maxbibnames=99, sorting=none]{biblatex}
%\addbibresource{mylibrary.bib}
\addbibresource{CRB.bib}

% Use 'disable' as a parameter to hide todos
\usepackage[]{todonotes}
\usepackage{framed}

% Prevent page breaks within paragraphs
% https://tex.stackexchange.com/questions/21983/how-to-avoid-page-breaks-inside-paragraphs
%\widowpenalties 1 10000

\usepackage{ragged2e}
\usepackage{titlepic}


\begin{document}
%\titlehead{USDA-APHIS Progress Report}
\title{An Update on the UOG Coconut Rhinoceros Beetle Project}
\titlepic{\includegraphics[width=.9\textwidth]{images/rhino_beetle_head}}
\author{Aubrey Moore\\University of Guam\\College of Natural and Applied Sciences}
\date{April 14, 2021}
\maketitle

\begin{footnotesize}
\url{https://github.com/aubreymoore/CRB-Project-Update-2021/raw/main/CRB-project-update.pdf}
\end{footnotesize}

\newpage{}
\tableofcontents{}

%\todo[inline]{clean crap from source}
%\listoftodos

%\newpage

%This brief is in response to a request for information on the status of the Coconut Rhinoceros Beetle (CRB) Project at UOG.
%
%The University of Guam College of Natural and Applied Sciences (UOG-CNAS) supports efforts to find solutions to invasive species problems on Guam and elsewhere in Micronesia by providing technical and scientific assistance, including applied research.

\newpage

\section{Background}

\subsection{CRB Biology and Invasion History}

\begin{figure}
	\centering
	\includegraphics[width=\linewidth]{images/crb_life_cycle2_cropped}
	\caption{Coconut rhinoceros beetle life cycle.}
	\label{fig:crblifecycle2}
\end{figure}

CRB is a large scarab beetle native to southeast Asia. Like all 
beetles, CRB has four life stages: egg, grub, pupa, and adult (Fig. \ref{fig:crblifecycle2}). Only the adult stage causes 
damage. Grubs feed only on decaying vegetation and do no harm. 
Adult males and females bore into the crowns of coconut palms and 
other palms to feed on sap. They do not feed on leaves, but they 
bore holes through developing leaves on their way to the white 
tissue at the interior of the crown. When these damaged leaves 
eventually emerge from the crown, they have v-shaped cuts in 
them, a distinctive sign of CRB damage (Fig. \ref{fig:dyingcoconuts}). 
Each adult feeds on sap for only a few days. It then leaves the 
crown to search for a breeding site. Palms may be killed if a CRB 
bores through the growing tip (the meristem). Mature palms are 
rarely killed at low CRB population levels. However, trees are 
killed when they are simultaneously attacked by many adults 
during a population outbreak such as the one we are currently 
experiencing on Guam.

CRB breeding sites can be found in any mass of decaying 
vegetation. Preferred sites are standing dead coconut stems and 
fallen coconut logs and fronds. But piles of anything with a high 
concentration of decaying vegetation can be used as a breeding 
site including green-waste, dead trees of any species, saw dust, 
and manure. CRB breeding sites have even been found in 
commercially bagged soil purchased from a local hardware store \cite{moore_movement_2016}. An active breeding site will contain all CRB life stages. 
Adults locate breeding sites by sniffing out a chemical signal 
referred to as an aggregation pheromone. This pheromone has been 
synthesized and is commercially available \cite{hallett_hallett_1995}. 

A female rhino beetle lays about 100 eggs during her lifetime. 
Assuming a 50\% sex ratio and 100\% survival, there will be a 
population increase of 5,000\% during each generation. Thus 
population explosions may occur when abundant potential breeding 
sites are available in the form of rotting vegetation following 
destruction in the wake of a typhoon, large scale land clearing, 
or war. Large numbers of CRB adults generated by a population 
explosion may result in large numbers of palms being killed. The 
dead standing trunks soon become ideal breeding sites which 
generate even higher numbers of adults. This positive feedback 
cycle will end when the rhino beetles run out of food, meaning 
when most of the palms have been killed and rotted away.

CRB invaded islands in the Pacific and Indian oceans during two 
waves of movement. The first wave started in 1909 when 
CRB was accidentally transported to from Sri Lanka to Samoa with 
shipment of rubber tree seedlings and it ended during the 1970s \cite{catley_coconut_1969}. All of the CRB range expansion during this period was south of 
the equator except for the invasion of the Ryuku Islands (Japan) 
starting in 1921 \cite{oshiroBiologicalStudiesCoconut1980} and invasion of the Palau Islands in about 
1942 \cite{catley_coconut_1969}. In Palau, there was a population explosion of rhino 
beetles because WWII activities created abundant breeding sites. 
This resulted in about 50\% coconut palm mortality overall, and 
total loss of coconut palms on some of the smaller islands \cite{gressitt_coconut_1953}.

The second wave of CRB invasions started in 2007 with discovery 
of CRB on Guam, followed by invasion of Oahu (Hawaii), Port 
Morseby (Papua New Guinea), Guadalcanal, Savo and Malaita 
(Solomon Islands), and Rota (Commonwealth of the Northern Mariana 
Islands). Beetles in the second wave of invasions are genetically 
different from those in the first wave \cite{marshall_new_2017-1} and these are being 
referred to as the \textbf{Guam biotype} or \textbf{CRB-G} for short.

\begin{figure}
	\centering
	\includegraphics[width=0.7\linewidth]{images/dying_coconuts}
	\caption{Coconut palms dying after a severe attack from coconut rhinoceros beetles adults.}
	\label{fig:dyingcoconuts}
\end{figure}

\begin{figure}
	\centering
	\includegraphics[width=\linewidth]{images/crb_dist}
	\caption{Geographic distribution coconut rhinoceros beetle, \textit{Oryctes rhinoceros}. Green markers: native range; Brown markers: first detected in the 20th century; Red markers: first detected in the 21st century; Open circle: population includes CRB-G biotype; Filled circle: population is exclusively CRB-G biotype. An interactive version of this map is available online at \url{http://aubreymoore.github.io/crbdist/mymap.html}.}
	\label{fig:crbdist}
\end{figure}

\clearpage

\subsection{Guam CRB Eradication Program}

CRB was first detected on Guam in Lower Tumon on 
September 11, 2007. An island-wide 
survey, completed within two weeks, located CRB grubs and adults 
and damage symptoms only in Lower Tumon and at the adjacent 
Faifai Beach, an area totaling less than 1,000 acres \cite{smith_early_2008}. Based 
on this information it was decided that eradication would be 
attempted. Here, I am using the word eradication in the proper 
sense, meaning killing every single CRB on the island. 

In theory, a CRB population can be eradication from an island by 
locating and destroying all active breeding sites and ensuring 
that the arrival pathway is blocked to prevent re-infestation. In 
practice, CRB eradication is difficult. There have been several 
CRB eradication attempts, but only one of these was successful. 
CRB was eradicated from Niuatoputapu Island, also known as Keppel 
Island, a tiny outer island of Tonga, only 16 square kilometers 
in area. Eradication was accomplished by a sanitation program 
which lasted 9 years following first detection of CRB in 1921 \cite{catley_coconut_1969}.

The Guam CRB Eradication Project was a joint effort involving the 
United States Department of Agriculture, the Guam Department of 
Agriculture and the University of Guam. Financial support came 
from the United States Department of Agriculture, the United 
States Forest Service and the Legislature of Guam. The project 
used several tactics aimed at wiping out the CRB population: 
quarantine, sanitation, trapping, and chemical control. These are 
explained below.

The opportunity to eradicate CRB from Guam was lost when the 
infestation spread from the Tumon Bay area to breeding sites on 
other parts of the island prior to 2010. Most breeding sites are 
currently inaccessible for application of eradication tactics, 
being in the deep jungle and/or on military property which 
includes about one third of the island.

\subsubsection{Quarantine} 

The Guam Department of Agriculture drew a line around the CRB 
infestation area and required inspection and/or treatment of any 
dead vegetation being transport to other parts of the island. The 
quarantine had to be expanded several times. By 2010, all parts 
of Guam were infested by CRB.

\subsubsection{Sanitation} 

Sanitation is the most important tactic in any CRB eradication 
project. The target is to find and destroy all breeding sites 
before adults develop, thus halting reproduction and preventing 
all damage. The Guam eradication program employed 4 detector dogs 
trained to sniff out rhino beetle grubs.

\subsubsection{Trapping} 

At the start of the eradication project, we were advised that the 
adult population could be annihilated using the commercially 
available aggregation pheromone as a lure. To the contrary, we 
soon learned that the traps were ineffective for population 
suppression when new damage appeared in mass trapping areas. 
However, an island-wide trapping network of about 2,000 traps was 
useful for monitoring the spread and growth of the CRB 
population. Island-wide trapping was discontinued when Typhoon 
Dolphin visited the island in May 2015. 

A lot of work was done to improve pheromone traps. Our best 
pheromone trap design catches about 20 times as many beetles as 
the original standard trap we started with. However, even these 
traps catch only about 10\% of the adult beetles active in mass 
trapping areas: not high enough for effective population 
suppression under current conditions. During our CRB trap 
improvement project, we discovered that local fishermen were 
using a small fish gill net called tekken to capture CRB adults 
emerging from compost piles. This has become a useful tool for 
managing CRB. Tekken captures about 65\% of adults attempting to 
leave infested compost or green-waste piles. What may be more 
important is that these same piles become attractive to incoming 
adults which are also trapped. 

\subsubsection{Chemical control} 

Individual palms can be protected from CRB attacks by 
prophylactic insecticide application. But this is very expensive. 
A row of 33 severely damaged coconut palms at the University of 
Guam Agricultural Experiment Station in Yigo were nursed back to 
apparent perfect health by spraying their crowns with the 
insecticide cypermethrin on a biweekly schedule. It took 15 
months of treatment before all damaged fronds were replaced by 
healthy ones.

\subsection{Guam CRB Biological Control Program}

In their native environment insect populations are suppressed by 
natural enemies which include parasites, predators and pathogens. 
When alien insects invade islands they escape control by the natural 
enemies they left behind, resulting in damaging population explosions. Biological 
control programs introduce biocontrol agents which specifically 
target invasive species. There are many examples of successful 
biological control programs on Guam where invasive species 
populations are maintained at low levels by purposefully 
introduced biocontrol agents. 

The Guam CRB Biological Control Program was launched following 
failure of the eradication attempt. There are two widely used 
biocontrol agents for CRB, a virus called \textit{Oryctes rhinoceros} 
nudivirus (OrNV) and a fungus called \textit{Metarhizium majus}. Both of 
these pathogens attack only rhino beetles and pose no risk to 
other organisms.

\subsubsection{Biocontrol Agent 1: \textit{Oryctes rhinoceros} Nudivirus (OrNV)}

Shortly after its discovery in Malaysia in 1963 \cite{huger_oryctes_2005-1}, OrNV 
(\textit{Oryctes rhinoceros} nudivirus), was released as a biocontrol 
agent on Pacific islands which had been invaded by CRB. The 
release technique, referred to as auto-dissemination, is very 
simple: adult rhino beetles are infected with the virus and then 
released. The virus quickly spreads throughout the CRB population 
and persists in the environment indefinitely. Wherever the virus 
was released, there was a drastic decline of the beetle 
populations followed by a conspicuous recovery of the badly 
damaged coconut stands. After OrNV was released in Palau, CRB 
damage symptoms disappeared almost entirely. We expected the same 
results on Guam.

OrNV was sourced from AgResearch New Zealand, a NZ government 
research lab which maintains a collection of OrNV isolates. One 
isolate was imported and released on Guam under conditions of 
permits from USDA-APHIS and the Guam Department of Agriculture. 
Unexpectedly, we observed no response to treatment at field 
release sites. Subsequent laboratory tests showed that the Guam 
CRB population is resistant to the OrNV isolates available from 
AgResearch. This discovery lead to the realization that we are 
dealing with a genetically distinct CRB which is resistant to 
biological control by OrNV, which has been named CRB-G \cite{marshall_new_2017-1}. 
Although the \textbf{G} refers to Guam, genetic evidence shows that this 
biotype evolved thousands of years ago, prior to its arrival on 
Guam. 

As previously mentioned, all CRB invasions of Pacific islands in 
the 21st century involve CRB-G. The current resurgence of CRB 
damage in Palau is thought to be caused by a recent invasion of 
the islands by CRB-G rather than by a different biotype which has 
been there since WWII. 

\subsubsection{Biocontrol Agent 2: \textit{Metarhizium majus}}

\textit{Metarhizium majus} is a fungus which acts as a pathogen in 
rhinoceros beetles. Spores of \textit{M. majus} are produced in a 
laboratory using corn or rice as a substrate and the resulting 
material is applied to breeding sites as a bio-insecticide. 

\textit{M. majus} was imported from the Philippine Coconut Authority and 
released on Guam under conditions of permits from USDA-APHIS and 
the Guam Department of Agriculture. There were no observations of 
CRB grubs or adults infected with \textit{M. majus} prior to release of 
the fungus. An extensive post-release survey showed that between 
10\% and 38\% of field collected CRB died from M. majus infection 
within 21 days after collection. \textit{M. majus} has established on Guam 
and it is often found in untreated breeding sites. However, this 
biocontrol agent did not suppressed populations enough to prevent 
the current outbreak. 

\subsection{Typhoon Dolphin Triggers a CRB Population Explosion}

When we thought that the Guam CRB problem could not get worse, it 
did. Typhoon Dolphin triggered the current CRB outbreak we are 
now experiencing.

Typhoon Dolphin visited Guam in May 2015. It was not a very 
strong typhoon by Guam standards, but it was the first one in 
more than a decade and it created a lot more damage than 
expected. Abundant piles of decaying vegetation became CRB 
breeding sites. Some of these new breeding sites were in villages 
were they could be managed. But most were inaccessible: in 
jungles and/or on military land. Within a few months, massive 
numbers of adults were emerging from breeding sites and severely 
attacking palms which started to die. Prior to Dolphin, we saw 
some heavily damages palms, but very few dead ones. Once a palm 
is killed, its dead standing trunk becomes an excellent breeding 
site which eventually produces even more adults resulting in a 
self-sustaining outbreak such as the one we are experiencing.

\subsection{Where do we go from here?}

If we do not control the current rhino beetle outbreak on Guam, 
it will only end when the beetles run out of food. Which means 
most of Guam's palm trees will be killed, as happened in Palau 
after WWII. If current CRB-G outbreaks in the Pacific cannot be 
suppressed, it is only a matter of time until this biotype 
invades other islands through accidental transport. If CRB-G 
reaches atolls where the coconut palm is the tree of life this 
will be a human tragedy, possibly displacing islanders to larger 
population centers.

At the 2016 International Congress of Entomology, the USDA 
sponsored a meeting to plan a regional response to CRB-G. 
Pacific-based entomologists with extensive experience working 
with CRB agreed that our best bet for stopping CRB-G outbreaks is 
to find an effective biocontrol agent for CRB-G. Most likely this 
will be an isolate of OrNV which is highly pathogenic for CRB-G. 

Finding an OrNV isolate which can be used as an effective biocontrol agent for CRB-G is
currently the top priority of the UOG CRB project.

\newpage

\section{Current Funding}

Since CRB was first detected on Guam in 2007, UOG-CNAS has continually collaborated with Guam DoAg and local USDA-APHIS staff with grant funding from various sources including:
\begin{itemize}
	\item USDA Animal and Plant Inspection Service (USDA-APHIS)
	\item US Forest Service (USFS)
	\item Department of the Interior - Office of Insular Affairs (DOI-OIA)
	\item Guam Legislature
	\item Western Integrated Pest Management Center
\end{itemize}

Until recently, most financial support for UOG CRB project came from USDA-APHIS Plant Protection Act funds (previously Farm Bill funds). However, despite the fact that CRB damage on Guam is much more severe than that on Oahu, \href{https://github.com/aubreymoore/wiki/wiki/USDA-APHIS-Farm-Bill-and-Plant-Protection-Act-Funding-for-Coconut-Rhinoceros-Beetle-in-Guam-and-Hawaii}{APHIS financial support for CRB work on Guam and Hawaii has become extremely disproportionate}. During FY2020-21, Guam Department of Agriculture was granted \$140,000 from PPA funds for CRB work and two UOG proposals were rejected. During this same period, Hawaii Department of Agriculture received \$5,342,066 PPA funding for CRB work.

Here is a list of grants currently supporting the project:

\bigskip

\begin{tabular}{ll>{\raggedright}p{2in}cr}
	\hline 
	Funding Source & Amount & Title &  Start date & End date \\ 
	\hline 
	USDA-APHIS & \$200.000 &Biological control of Coconut Rhinoceros Beetle Biotype G on Guam & 2019-08-08 & 2021-08-07 \\ 
	\hline
	DOI-OIA & \$239,994 & Establishment of Self-sustaining Biological Control for Coconut Rinoceros Beetle Biotype G in Micronesia & 2020-05-14 & 2023-09-03 \\
	\hline
	USFS & \$98,240 & Establishment of Self-sustaining Biological Control of Coconut Rhinoceros Beetle Biotype G in Micronesia & 2020-06-17 & 2021-05-30 \\
	\hline
    USFS & \$23,000 & Improving Coconut Rhinoceros Beetle Breeding Site Detection Using Harmonic Radar & 2020-06-17 & 2021-05-30 \\	
	\hline 
\end{tabular} 

\section{Current Staffing}

\begin{description}
	\item[PI:] Dr. Aubrey Moore, Entomologist
	\item[PostDoc:] Dr. James Grasela, Insect Pathologist	
	\item[Technician:] Christian Cayanan	
\end{description}

\section{Recent Progress}

The following subsections were extracted from the latest USDA-APHIS progress report for the project \cite{mooreUSDAAPHISPPA2019ProgressReport2021}. Note that this report and almost all documents generated by the project are available online.

%\clearpage
%\textbf{Notes for the Reader}
%\begin{itemize}	
%	\item This progress report covers August 2020 through March 2021.	
%	\item Objectives and methods, as stated in the approved work plan \cite{mooreWorkPlanAPHISPPA2020} are presented in a frame at the start of each section. 	
%	\item The University of Guam CRB biological control project is a long-term effort supported by multiple short-term grants. Some of the activities reported here have been partially funded by sources external to the FY19 APHIS-PPA CRB Biocontrol grant.  	
%\end{itemize}
%
%


\begin{framed}
\subsection{Objective 1: CRB Control}

The primary objective is to find an \textit{Oryctes rhinoceros} nudivirus (OrNV) isolate which can be used as a highly effective biological control agent for long-term suppression of CRB-G populations. As soon as laboratory studies indicate discovery of an OrNV isolate which is a potential biological control agent for CRB, we will multiply the virus \textit{in vivo} and initiate field releases under the conditions of an existing USDA-APHIS permit.
\end{framed}

We have attained little progress towards completing our primary objective, which is to find an isolate of OrNV which is a promising candidate for biocontrol of CRB-G, because of COVID-19 travel restrictions and technical problems with our bioassays. Impediments towards progress are discussed in detail in Section \ref{impediments}.

\begin{framed}
\subsubsection{Regional Collaboration}

Work will continue with colleagues at AgResearch New Zealand, the Secretariat of the Pacific Community (SPC), Tokyo University, the University of Hawaii and others to put together a regional collaboration with the objective of finding an effective biocontrol agent for CRB-G.
\end{framed}

Project staff promoted Pacific-wide collaboration among people trying to find solutions to the CRB problem by helping to organize professional meetings and by providing online tools to facilitate sharing of scientific and technical information.

\paragraph{CRB Action Group Meeting; December 9, 2020}
This meeting was run as a Zoom webinar hosted by the University of Guam. A recording is available online.\cite{mooreVideoRecordingCRBG2021}

\subparagraph{Agenda and Presentations:}

\begin{itemize}
	\item Regional Reports
	\item \textbf{Aubrey Moore}: Automated roadside video surveys for detecting and monitoring CRB damage to coconut palms. \cite{mooreAutomatedRoadsideVideo2020}
	\item \textbf{Sarah Mansfield}: Bioassays with OrNV: Progress and pitfalls.
	\item \textbf{Madoka Nakai and Shunsuke Tanaka}: Characterization of a Palauan isolate of \textit{Oryctes rhinoceros} nudivirus (OrNV).
	\item \textbf{Kayvan Etebari}: Transcriptomic responses of different geographical populations of coconut rhinoceros beetles to \textit{Oryctes rhinoceros} nudivirus (OrNV) infection.
	\item Open Discussion	
\end{itemize}


\paragraph{CRB Action Group Meeting; March 17, 2021}
This meeting was run as a Zoom webinar hosted by the University of Guam. A recording is available online \cite{mooreVideoRecordingCRBG2020}.

\subparagraph{Agenda and Presentations:}
\begin{itemize}
	\item Regional Reports
	\item \textbf{Sulav Paudel}: Review paper on biological control of CRB
	\item \textbf{Mark Ero}: Evaluation of CRB-G pheromone from the Solomon Islands
	\item \textbf{Lastus Kuniata}: Insecticide work in young oil palm in Solomon Islands
	\item \textbf{Katayo Sagata}: Evaluation of integrated management options in an oil palm cropping system in PNG
	\item Open Discussion	
\end{itemize}

\paragraph{Rota CRB Eradication Program}
The PI made a presentation in a webinar organized by Department of the Interior - Office of Insular Affairs (DOI-OIA) entitled \textit{CRB Biology: Know Your Enemy} \cite{mooreCRBBiologyKnow2021}. This webinar will be converted into a podcast and published online by DOI-OIA which is supporting the Rota CRB Eradication Project with a \$250,000 grant.

\paragraph{LISTSERV}
The PI established a LISTSERV to facilitate email discussions among members of the CRB Action Group.\cite{mooreOnlineEmailDiscussion2021}

\paragraph{Online CRB Reference Library}
Project staff maintain an online reference library of technical and scientific information about coconut rhinoceros beetle.\cite{mooreOnlineReferenceLibrary2021}


\begin{framed}
\subsubsection{Foreign Exploration for an Effective Biocontrol Agent for CRB-G}

Foreign exploration in search of a microbial biocontrol agent for CRB-G is already underway. During January, 2017, Moore, Iriarte and Marshall collected an isolate if OrNV from a CRB-G population in Negros Island, Philippines. Laboratory bioassays indicate that this isolate is not a good candidate for biocontrol.

We are currently performing laboratory bioassays to evaluate two novel isolates obtained from AgResearch New Zealand. In addition we are attempting to isolate OrNV from CRB adults collected in Taiwan. This population was targeted because Dr. Shizu Watanabe, University of Hawaii, reported an 82\% OrNV infection rate in CRB-G collected from this island.

Our next target population is CRB-G found on the southern islands of Japan. We plan to collaborate with Dr. Madoka Nakai, Tokyo University of Agriculture and Technology, to obtain CRB-G/OrNV specimens from these islands.

%\subsubsection{Methods}
%
%\begin{itemize}
%	\item Subsamples of CRB collected during foreign exploration will be shipped to AgResearch New Zealand to determine CRB biotype and to isolate OrNV
%	\item OrNV isolates collected during foreign exploration will be tested in the insect pathology lab at UOG using standard bioassay protocols.
%	\item The PI and insect pathology post-doc will travel to Japan to collect CRB-G and OrNV. 
%\end{itemize}
\end{framed}

%\subsubsection{Progress}

The project applied for and was granted USDA-APHIS permits to import \textit{Oryctes rhinoceros} \cite{usda-aphis_crb_2019} and \textit{Oryctes rhinoceros} nudivirus (OrNV) \cite{usda-aphisImportPermitOrNV2020}.

No progress was made on foreign exploration for OrNV isolates as potential biocontrol agents for CRB during the period covered by this report because of COVID-19 travel restrictions. See section \ref{covid} for details.

\begin{framed}
\subsubsection{Establish Lab Colonies of CRB-G and CRB-S}

We will establish sustainable laboratory colonies of CRB-G and virus susceptible beetles (CRB-S) as a source of healthy beetles for bioassays.

Note: Establishment of a CRB-S colony is contingent on receiving a USDA-APHIS import permit to import live coconut rhinoceros beetles. I requested a permit on March 19, 2019 (Application number P526-190319-001) to replace a previous permit, P526P-11-01844, which I accidentally allowed to lapse into oblivion after only one shipment.

If we are allowed to import CRB-S, this will allow us to do comparative studies to:

\begin{itemize}
\item Measure difference in susceptibility to OrNV isolates. (Resistance of CRB-G to OrNV has not yet actually been proven by comparative bioassays.)
\item Test for behavioral differences. (It has been hypothesized that the aggregation pheromone, oryctalure, is less attractive to CRB-G than CRB-S.)
\end{itemize}

%\subsubsection{Methods}
%
%\begin{itemize}
%	\item Larvae and adults will be reared individually in Mason jars enclosed by metal caps. Larvae will be fed a store-bought steer manure/soil blend on which we have reared CRB larvae for many years. Adults will be bedded in peat moss and fed banana slices weekly.
%	\item Mason jars will be placed in environmental cabinets set at 30 deg. C, 80\% RH and 12 h photoperiod.
%	\item Detailed records for each individual beetle will be stored in an existing online laboratory information management system (LIMS). These data will be made available to USDA-APHIS.
%\end{itemize}
\end{framed}

%\subsubsection{Progress}



The project applied for and was granted a USDA-APHIS permit to import \textit{Oryctes rhinoceros}, Biotype CRB-S \cite{usda-aphis_crb_2019}, \cite{moore_additional_2019}.

No progress was made on establishing laboratory colonies for CRB-G and CRB-S biotypes during the period covered by this report because of COVID-19 travel restrictions and University of Guam closures. See section \ref{covid} and section \ref{stayathome} for details.


\begin{framed}
\subsection{Objective 2: Establish a Sustainable Coconut Palm Health Monitoring System}

The CRB-G outbreak on Guam is currently unmonitored on an island-wide basis. An island-wide pheromone trapping system, using about 1500 traps, was operated by the University of Guam from 2008 to 2014. This monitoring system was transferred to the Guam Department of Agriculture which abandoned the effort at the end of February, 2016.

Currently, many coconut palms are being killed by CRB-G. But, in the absence of a monitoring system, we do not have an estimate of tree mortality or whether or not the damage is increasing or decreasing. Clearly, establishment of a monitoring system is necessary if we want to evaluate success of the proposed biocontrol project, or any other mitigation efforts.

Rather than re-establish a trapping survey, we intend to establish a monitoring system to track temporal and spatial changes in the extent of CRB damage to Guam's coconut palms. Damage symptoms such as v-shaped cuts to fronds, bore holes, and dead standing coconut palm stems are readily observed during roadside surveys. Survey data will be collected using a digital video camera mounted on a truck. Initially, video images of coconut palm damage by CRB-G will be detected, classified and tagged by a technician. When a large number of images have been tagged, these will be used to train a fully automated CRB damage detection and monitoring system. This automated system may be useful as an early detection device for CRB. Roadside surveys on Guam will be performed bimonthly.
%
%\subsubsection{Methods}
%
%\begin{itemize}
%	\item A protocol will be developed to perform roadside surveys of CRB damage. Damage will be recorded using videos recorded by a vehicle-mounted Olympus TG-5 camera. This camera records GPS coordinates.
%	\item Videos will be tagged using the open source Computer Vision Annotation Tool (CVAT).
%	\item An object detector which locates and classifies CRB damage in video recordings will be trained using annotated videos from the previous step. We intend to use the TensorFlow implementation of the Faster R-CNN Deep Learning model. Training a CRB damage detector using deep learning requires use of a computer with specialized software (TensorFlow) and specialized hardware (a graphics processing unit (GPU)). Instead of purchasing a physical machine we will rent a virtual machine designed specifically for this application from Amazon Web Services.
%	\item Results from the trained object detector will be evaluated using the human annotated videos.
%	\item We will develop an automated processing system which takes roadside videos as input and generates CRB damage maps as output.
%\end{itemize}
\end{framed}

%\subsubsection{Progress}

Bimonthly automated roadside video surveys for CRB damage are now operational on Guam and the system has been tested on Rota. Videos recorded with a smart phone attached to a vehicle are analyzed using custom-designed artificial intelligence software which recognizes coconut palms and measures CRB damage. A nontechnical description of the survey method is given in the next section.

A presentation on this new CRB survey methodology was made at the December 9 2020 meeting of the CRB-G Action Group conducted as a Zoom webinar \cite{mooreVideoRecordingCRBG2020}, \cite{mooreAutomatedRoadsideVideo2020}.

\paragraph{Guam Roadside Video Survey 1}

The following four images were extracted from a draft of the University of Guam's Western Pacific Tropical Research Center impact report for 2020. They provide a nontechnical overview of the new automated roadside video survey for CRB damage and results from the first Guam survey in October, 2020.

\begin{figure}[h]
	\centering
	\includegraphics[width=1\linewidth]{images/impact-report07.png}
	\caption{Feature article in the University of Guam's Western Pacific Tropical Research Center impact report for 2020.}
	\label{fig:roadside1-1}
\end{figure}

\begin{figure}[h]
	\centering
	\includegraphics[width=1\linewidth]{images/impact-report08.png}
	\caption{[Continued] Feature article in the University of Guam's Western Pacific Tropical Research Center impact report for 2020.}
	\label{fig:roadside1-2}
\end{figure}

\begin{figure}[h]
	\centering
	\includegraphics[width=1\linewidth]{images/impact-report09.png}
	\caption{[Continued] Feature article in the University of Guam's Western Pacific Tropical Research Center impact report for 2020. This interactive web map is publicly available at: \url{https://aubreymoore.github.io/new-crb-damage-map}}
	\label{fig:roadside1-3}
\end{figure}

\begin{figure}[h]
	\centering
	\includegraphics[width=1\linewidth]{images/impact-report10.png}
	\caption{[Continued] Feature article in the University of Guam's Western Pacific Tropical Research Center impact report for 2020.}
	\label{fig:roadside1-4}
\end{figure}

\clearpage
\paragraph{Guam Roadside Video Survey 2}

The proportion of coconut palms damaged by CRB increased significantly from 19.2\% in
October 2020 to 21.5\% in December 2020 (p < 0.001; Fisher's exact test).

\begin{figure}[h]
	\centering
	\includegraphics[width=1\linewidth]{images/crb-webmap-2020-12.png}
	\caption{Screenshot of an interactive web map of results from a roadside video survey of
		CRB damage on Guam in December 2020 \url{https://aubreymoore.github.io/Guam-CRB-damage-map-2020-12/webmap/v1/}.}
	\label{fig:guam02}
\end{figure}


\clearpage
\paragraph{Rota Roadside Video Survey 1}

Rota was invaded by CRB in 2017 and eradication efforts by Rota Department of Land and Natural Resources have successfully kept the population at a very low level, although the population has begun to spread to new areas of the island. In October 2020, a smart phone and associated equipment was sent to Rota-DLNR so that they could do an initial roadside video survey in support of their CRB control efforts. In addition to the equipment, a survey setup guide and  \cite{aubreymooreSetAutomatedRoadside2020} and a setup video \cite{mooreYouTubeVideoMounting2020} were prepared and sent.

The survey was performed by Mark Manglona, Rota-DLNR and the phone containing videos from the survey was returned to the University of Guam.  Videos were analyzed using the workflow developed for the Guam surveys. The resulting web map contained many false positives for CRB damage, but there is one hit which shows a classic v-shaped cut probably caused by CRB. For convenience, data for this hit (images, date, location) were documented as an iNaturalist observation (Figure \ref{fig:rota-inat-obs}). If this v-shaped cut was caused by CRB, there will be a bore hole. Rota-DLNR located the damaged palm but did not find a bore hole.  Therefor the damage was not caused by CRB.


\begin{figure}[h]
	\centering
	\includegraphics[width=1\linewidth]{images/Rota-iNat-obs}
	\caption{Screenshot of an iNaturalist observation documenting possible coconut rhinoceros beetle damage detected during a roadside video survey performed by Rota DLNR. \url{https://www.inaturalist.org/observations/69534809}.}
	\label{fig:rota-inat-obs}
\end{figure}

\newpage

\section{Impediments to Progress in the Guam CRB Biocontrol Program}
\label{impediments}
Significant progress was made on the Damage Survey and Regional Collaboration
components of this grant. However, progress on the Biological Control component is
lagging because of travel restrictions, stay at home orders, and a major technical problem.

The technical problem was PCR positive tests for OrNV in field collected CRB on Guam. Bioassays aimed at finding effective OrNV biocontrol candidates were stopped to investigate this problem. See details below. 

\subsection{Delays Caused by Travel Restrictions}
\label{covid}

The original work plan included foreign travel to collect novel OrNV isolates for evaluation
as biocontrol agents and to collect virus susceptible CRB biotypes for comparative bioassays.
Our first planned collecting trip was to visit American Samoa in December 2019. This trip
was canceled at the last minute because of a measles outbreak. Our second attempt, in
March 2020 was also canceled at the last minute, because of COVID-19 travel restrictions.

\subsection{Delays Caused by COVID-19 Stay-at-Home Orders and University Closures}
\label{stayathome}
Progress was further delayed by Government of Guam stay at home orders. The University
of Guam was officially closed from March 20 to May 10 2020 and again from August 16 2020
to January 15 2021.

\subsection{Delays Caused by Detection of OrNV in CRB Collected on Guam}

Our lab currently uses CRB-G adults collected from pheromone traps as test animals
in bioassays evaluating OrNV isolates as biocontrol agents under the assumption that the
Guam beetle population contains only the CRB-G biotype and is free from OrNV infection.
In 2019 we gained the capacity to perform PCR in our lab began testing these assumptions.
PCR results indicated that field-collected beetles were all CRB-G, but 18\% of these tested
positive for OrNV.

Based on these results, the PI decided to suspend bioassays until we had conclusive evidence of OrNV infection in the Guam CRB-G population. An experimental plan \cite{mooreExperimentalPlanDetermining2020} was
developed and executed. One hundred beetles were collected from each of two trapping sites
(Leo Palace Resort in Southern Guam and the UOG Ag. Expt. Stn. in northern Guam).
Gut samples were obtained from these beetles and tested using PCR in our lab and also in
Sean Marshall's lab at AgResearch New Zealand. In PCR results from both labs all beetles tested positive for CRB-G biotype and negative for OrNV infection \cite{graselaInvestigationDeterminePresence2020}. We suspect that
previous OrNV positive tests were the result of lab contamination (not false positives).
During the hiatus in OrNV bioassays, we re-examined our bioassay methodology and have
decided to make major changes before moving forward:

\paragraph{Re-establishment of a CRB rearing program.} High variance among results from bioassay
replicates and high mortality rates are a serious impediment to progress in finding an
OrNV isolate which can be used as an effective biocontrol agent for CRB-G. Beetles
collected from pheromone traps are not ideal test insects because they vary in age and
many are infected with \textit{Metarhizium majus}.
We intend to re-establish the Guam CRB rearing program to supply healthy, standardized test insects. Insects will be reared individually in Mason jars and a detailed record
will be maintained for each individual. Larvae will be fed a diet of heat-sterilized CRB
breeding site material from dead standing coconut trunks. The CRB-G lab colony will
be initiated with surface-sterilized eggs and isofemale lineages will be maintained.
Because the life cycle of the coconut rhinoceros beetle is about 9 months, there will
be a lag time of about one year before the rearing program is fully operational and bioassays can resume.

\paragraph{Measurement of sublethal impacts of OrNV infection.} The literature indicates
that reduction in damage to coconut palms after release of OrNV may be the result of
sublethal impacts on the population rather than mortality. These impacts may include
reduction in fecundity, feeding, and flight. Bioassays which measure only mortality may
reject promising biocontrol candidates.
We already indirectly track feeding by weighing beetles during bioassays but will also
include egg counts in future observations so that we can measure changes in fecundity. We are also considering using flight mills to measure flight capability.

\paragraph{Acquisition of virus susceptible biotypes.} One or more lab colonies of virus-susceptible biotypes are required for comparative bioassays. We
originally planned to source non-CRB-G beetles during foreign exploration for OrNV
isolates, but this has not happened because of COVID-19 travel restrictions.
Our new plan is to import gravid females from places such as Palau, where one or more
virus-susceptible biotypes are known to exist, in addition to CRB-G. Eggs from these
females will be used to establish isofemale lineages and these lines will be genotyped
using DNA extracted from the imported females.

\clearpage
\printbibliography[heading=bibintoc]

\end{document}
